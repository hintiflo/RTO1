\documentclass{article}

	\usepackage[utf8]{inputenc}
  % \usepackage[latin1]{inputenc}
	\usepackage[ngerman,naustrian]{babel}
	\usepackage{lmodern}
	\usepackage[T1]{fontenc}
	\usepackage{ulem}
	\usepackage{here}
	\usepackage[pdftex]{graphicx}
	\usepackage{amsthm}
	\usepackage{gensymb}
	\usepackage{fancyhdr}
	\usepackage[left=20mm,top=25mm,bottom=25mm,right=20mm,headheight=15mm,headsep=10mm,footskip=10mm]{geometry}
	\usepackage{longtable}
	\usepackage{hhline}
	\usepackage[table]{xcolor}
	\usepackage{amsfonts}
	\usepackage{amssymb}
	\usepackage{amsmath}
	\usepackage{mathcomp}
	\usepackage{tabularx}
	\usepackage{multicol}
	\usepackage{graphicx}
  % \usepackage{listings}
	\usepackage{minted}
\renewcommand{\familydefault}{\sfdefault}
		% \sffamily

\newcommand{\bild}[3]{\begin{figure}[h!]		\begin{center}			\includegraphics[#3]{#1}			\caption{#2}		\end{center}	\end{figure}}

\begin{document}

%\pagestyle{empty}
\begin{titlepage}
	\begin{center}
		{\large{FH OÖ - Hagenberg \\ embedded systems design}\\\vspace*{4cm}}
		\small{RTO1 UE}\\
		\textbf{WS 2020}\\\vspace*{2cm}
		\Huge{\textbf{Protokoll}}\\\vspace*{1cm}
		\huge{Übung\,2: „Kochrezept“ eines Echtzeitbetriebssystems\\ \large{ Round Robin } } \vspace*{90mm}
		
		\small{Simon Steindl  S2010567025	\\
		Florian Hinterleitner S2010567014	\\
		}
	\end{center}
\end{titlepage}

% \tableofcontents
% \newpage
% \setcounter{page}{18}
% \setcounter{section}{3}

\section*{Schritt 1 – Erste Strukturen und Funktionen anlegen		}

\section*{Schritt 2 – einfacher Round-Robin Scheduler		}

\section*{Schritt 3 – fertiger Round Robin Scheduler		}

\section*{Schritt 4 – Erweiterung APOS Critical Region		}

der TaskSwitch wird nun mit jeder Erhöhung des SysTick-Counters ausgeführt.
In der Folge wird das Mandelbrot-Fraktal nicht mehr gezeichnet, da der TimeSlice=SysTick mit 1ms für diesen nun zu kurz ist.

\section*{Schritt 5 – Erweiterung APOS Delay		}

\section*{Schritt 6 – Optimierung Scheduler und Messergebnisse		}

\subsection{Debug-Unit}

\subsection{Laufzeit jedes Tasks}
	\begin{table}[h!]
		\begin{center}		
			\begin{tabular}{ l | l }	 % \hline
				Task & Laufzeit in ms \\ \hline
				Systick mit Mandelbrot	& 18.52s \\ \hline
				Systick ohne Mandelbrot	& 29.39 	\\ \hline
				Systick ohne GPIOs *)	& 29.38 \\ \hline
				Counter	& 6.135 \\ \hline
				Key& 4.895 \\ \hline
				LED	& 4.894 \\ \hline
				Watch	& 7.346 \\ \hline
				Poti	& 6.115 \\ \hline
				Mandelbrot	& 18.49s \\ %\hline
			\end{tabular}
			\caption{Laufzeiten des SysTicks, sowie der einzelnen Tasks}
		\end{center}		
	\end{table}
\subsection{Overhead (Zyklen, µs) der Messung}
Der Overhead, also das schalten der GPIOs, errechnet sich aus der Lauzeit des Systemzyklus mit und ohne*) GPIOs: 29.39ms - 29.38ms = 0.01ms ~= 10us. Der 8MHz-Quarz wird laut system-stm32f0xx.c per PLL auf eine SYSCLK von 48MHz hochgetaktet, somit entspricht der Overhead rund 480 Taktzyklen. Die Zeitdifferenz ist als grober Schätzwert zu betrachten, da die Auflösung in der 2ten Kommastelle schon sehr gering ausfällt. \\
	*) jedoch mit SysTick GPIOs, ohne die gar keine Messung möglich wäre

\subsection{Screenshots der Messungen }
% \bild{tryGPIO}{GPIO-Treppe zum korrekten Verkabeln und Zuordnen der LA-Kanäle zu Tasks}{width=\textwidth}
% \bild{02_mitMandelbrot}{mit Mandelbrot-Task}{width=\textwidth}
% \bild{03_ohneMandelbrot_time}{ohne Mandelbrot-Task}{width=\textwidth}
% \bild{03_ohneMandelbrot}{ohneMandelbrot}{width=\textwidth}


\section{Übungsaufgabe B – Reaktionsgeschwindigkeit bei Superloops}

% \subsubsection*{StringOps gtest XML output}
% \inputminted[autogobble]{C++}{../../UE02/src/APOS.c}
			% C:\Users\r.hif\RTO1\RTO1\UE02\src\APOS.c

\end{document} 